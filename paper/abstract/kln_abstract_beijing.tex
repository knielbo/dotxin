\documentclass{article}
\usepackage{mss}
\title{Disruptions and Persistent Dynamics in Cultural Expressions}
\author{{KRISTOFFER~L.~NIELBO}}
\address{Center for Humanities Computing Aarhus \& Interacting Minds Centre, Aarhus University, Denmark}
\begin{document}
\maketitle

\textbf{\textit{Abstract}}\\
Humans exhibit a species-unique capacity for long-term planning and future-oriented cognition. This `deep temporality' is so fundamental to human behavior, that is can be considered the hallmark of our symbolically mediated environmental interactions. Cultural systems of symbolic production show non-linear dynamics and long-range temporal dependencies that can present a challenge to existing methods in culture research. A valid understanding of culture therefore depends critically on selection of adequate methods and access to sufficient data. In this paper, we present a novel approach that combines latent lexical models and fractal analysis with information theoretical theoretical concepts in order to model disruptive dynamics in cultural systems. To illustrate the approach's application, we use two very different data sets: 1) the collected writings of single authors; and  2) sampled text from the ctext corpus. Results show that with enough data, our approach can identify disruptive dynamics and transitions in cultural systems. We argue that when combined with domain knowledge in language and culture, the approach can be used to validate a set of generic theoretical claims about cultural stability and change.\\

\medskip
\textbf{\textit{Keywords}}\\
Culture Dynamics, Culture Analytics, Adaptive Fractal Analysis, Information Theory.\\ 

\medskip
\textbf{\textit{Biography}}\\

KLN is a humanities researcher that has specialized in applications of quantitative methods and computational tools in analysis, interpretation and storage of cultural data. He has participated in a range of collaborative and interdisciplinary research projects involving researchers from the humanities, social sciences, health science, and natural sciences. KLN's research covers two areas of interest of which one is more recent (automated text analysis) and the other (modeling of cultural behavior) has followed him during his entire academic career. Both interests explore the cultural information space in new and innovative ways by combining cultural data and humanities theories with statistics, computer algorithms, and visualization.
\end{document}