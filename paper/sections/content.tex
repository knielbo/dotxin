%%% PART 1 %%%

\section{Background}

\subsection{\texttt{CTEXT} database}
{\nologo
\begin{frame}{\texttt{CTEXT} database}
\begin{columns}
	\begin{column}{.5\textwidth}
		Corpus of 96 Classical Chinese texts from the \texttt{CTEXT} database\\
			-- $\sim$ 1000 BCE - 200 CE\\
			-- $N$ tokens\\
		\medskip
		\emph{Shangshu} of 58 chapters:\\
		\medskip
		\begin{tabular}{l | c}
			Period & Chapters \\
			\hline
			Pre-Warring & 10 \\
			Warring & 16 \\
			Late Warring - Early Han & 7\\
			Han & 25 \\
		\end{tabular}
	\end{column}
	\begin{column}{.5\textwidth}
		\begin{figure}
			\centering
			\includegraphics[width=\textwidth]{../fig/graph_nx_community.png}
		\end{figure}
	\end{column}
\end{columns}
\blfootnote{Slingerland, E., Nichols, R., Nielbo, K., \& Logan, C. (2017). The Distant Reading of Religious Texts: A Big Data Approach to Mind-Body Concepts in Early China. Journal of the American Academy of Religion, 85(4), 985–1016.}
\blfootnote{Nichols, R., Slingerland, E., Nielbo, K., Bergeton, U., Logan, C., \& Kleinman, S. (2018). Modeling the Contested Relationship between Analects, Mencius, and Xunzi: Preliminary Evidence from a Machine-Learning Approach. The Journal of Asian Studies, 77(01), 19–57.}
\end{frame}
}

%%% PART 2 %%%

\section{Erroneous class}

\subsection{Classifier}
\begin{frame}{Text-age (mis-)classifier}
\setbeamercolor{coloredboxstuff}{fg=black,bg=white!80!yellow}
\begin{beamercolorbox}[wd=1\textwidth,sep=1em]{coloredboxstuff}
Train a simple and transparent learning model to explore the boundaries of age classification
\end{beamercolorbox}
	\begin{itemize}
		\item[-] compare age-central to age-peripheral chapters of the \emph{Shangshu}
		\item[-] \textcolor{red}{error semantics} of the \emph{Shangshu}
	\end{itemize}
\medskip
The probability of a document $d$ being in class $c$, $P(c \mid d)$ is computed as:
\begin{equation*}
P(c \mid d) \propto P(c) \prod_{i = 1}^{m}P(t_i \mid c) 
\end{equation*}
and the class of a document $d$ is computed as:\\
\begin{equation*}%Maximum A Posteriori Estimation
c_{MAP} = arg~max_{c \in \{c_1, c_2 \}} P(c \mid d)
\end{equation*}
\end{frame}

\subsection{Misclassification}
\begin{frame}
	\begin{figure}
		\centering
		\includegraphics[width=0.75\textwidth]{/home/knielbo/Documents/figures/covariance_doc_misclass.png}
		\caption{Documents covariance matrix for all chapters of the \emph{Shangshu}}
	\end{figure}
Three documents from the \emph{Late Warring - Early Han} and one from \emph{Han} are age ambiguous\\ 
\end{frame}

\subsection{Document distance}
\begin{frame}
\begin{figure}
	\centering
	\includegraphics[width=\textwidth]{/home/knielbo/Documents/figures/mclss_kld.png}
	\caption{Average distance* to all classes show that the \textcolor{red}{error class} is closer to the document than the \textcolor{green}{correct class}}
\end{figure}
The distance between documents $s^{(1)}$ and $s^{(2)}$:
\begin{equation*}
%D_{KL}(d_1 \mid d_2)  = \sum_{i = 1}^{n} d_{i1} \times \log_2 \frac{d_{i1}}{d_{i2}}
D_{KL}(s^{(1)} \mid s^{(2)}) = \sum_{i = 1}^{K} s_i^{(1)} \times \log_2 \frac{s_i^{(1)}}{s_i^{(2)}}
\end{equation*}
\end{frame}

\subsection{Erroneous features}
\begin{frame}
\begin{figure}
	\centering
	\includegraphics[width=\textwidth]{/home/knielbo/Documents/figures/mclss_mi.png}
	\caption{Features that collectively are most central for classification. Signal from feature one is sufficient to explain the error.}
\end{figure}
\end{frame}

%%% PART %%%

\section{Disruptive dynamics}

\begin{frame}{Disruptive age effect}
\begin{figure}
	\centering
	\includegraphics[width=.75\textwidth]{/home/knielbo/Documents/figures/shangshu_kld.png}
	\caption{Length normalized lexical density over time for the Shangshu, notice the change points around and laminar region during \emph{Late Warring - Early Han}}
\end{figure}
\end{frame}

\subsection{Document representation}
\begin{frame}{Dense document representation}
-- model semantic disruption as ``variation on a theme''$\Rightarrow$ use a simple Bayesian model to capture lexical semantics\\
\smallskip
-- model each document as a distribution on lexical topics (e.g., {\small$P_1 = [0~.09~.78~.11~.2]$)}, where each `topic' is a distribution on words, and compare document similarity as the distance between any two documents with $chapter-index$ $j$ and $k$:\\
\smallskip
\begin{equation*}
D_{KL} (s^{(j)} \mid s^{(k)}) = \sum_{i = 1}^{K} s_i^{(j)} \times \log_2 \frac{s_i^{(j)}}{s_i^{(k)}}
\end{equation*}
-- bracket concrete semantics ($\sim$ reduce interpretive load) and only compare relative entropy between documents on topics (``variation on a theme'')\\
\end{frame}

\subsection{Qualitative similarities}
\begin{frame}
\begin{figure}
	\centering
	\includegraphics[width=.75\textwidth]{../fig/diverge_matrix.png}
	\caption{Distance matrix indicate some similarities with the sparse model (Fig. 1), notice the disruptive effect of \emph{Late Warring - Early Han} centered on $i = 30$.}
\end{figure}
\end{frame}

{\nologo
\subsection{Novelty \& resonance}
\begin{frame}{Disruptive dynamics}
Compute disruption as a combination of resonance on novelty:
\medskip
$\mathbb{N}$ovelty over window $w$:
\begin{equation*}
	\mathbb{N}_w (j) =  \frac{1}{w} \sum_{d=1}^{w}  D_{KL} (s^{(j)} \mid s^{(j - d)})
\end{equation*}
with $\mathbb{T}$ransience:
\begin{equation*}
\mathbb{T}_w (j) =  \frac{1}{w} \sum_{d=1}^{w}  D_{KL} (s^{(j)} \mid s^{(j + d)})
\end{equation*}
for $\mathbb{R}$esonance
\begin{equation*}
\mathbb{R}_w (j) = \mathbb{N}_w (j) - \mathbb{T}_w (j)
\end{equation*}s
\blfootnote{Murdock, J., C. Allen, S. DeDeo. 2015. Exploration and Exploitation of Victorian Science in Darwin's Reading Notebooks. arXiv:1509.07175 .}
\blfootnote{Nielbo, K.L., M.L. Perner, C Larsen, J Nielsen, Laursen D. 2019. Automated Compositional Change Detection in Saxo Grammaticus' Gesta Danorum. hal-02084682}
\end{frame}
}

\subsection{Results}
\begin{frame}
\begin{figure}
	\centering
	\includegraphics[width=\textwidth]{../fig/signals.png}
	\caption{\emph{Shangshu} Chapters' $\mathbb{N}$ovelty, $\mathbb{T}$ransience, and $\mathbb{R}$esonance for $w = 3$}
\end{figure}
\begin{columns}
	\begin{column}{.5\textwidth}
		\begin{figure}
			\centering
			\includegraphics[width=\textwidth]{../fig/novelty_resonance.png}
			\caption{\emph{Shangshu} Chapters' $\mathbb{R}$esonance on $\mathbb{N}$ovelty for $w = 3$}
		\end{figure}
	\end{column}
	\begin{column}{.5\textwidth}
		\textcolor{blue}{\emph{in conclusion}}\\
		-- \emph{Late Warring - Early Han} display age class atypical behavior \\
		-- lexical density shows global minimum and laminar behavior during \emph{Late Warring - Early Han}\\ 
		-- two disruptive maxima are located in \emph{Late Warring - Early Han}\\
		\medskip
		-- saturation followed by innovation\\
		-- class-dependent findings (study 1) confirmed by class-independent model (study 2)\\
	\end{column}
\end{columns}
\end{frame}