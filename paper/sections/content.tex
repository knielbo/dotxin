% SHORTCUTS
	%\fcolorbox{black}{yellow}{}

%%%%% PRELUDE %%%%%
%\section{On Artificial Intelligence}


%\subsection{Food to media hype}
%\begin{frame}{OpenAI's transformer-based model}

%\setbeamercolor{coloredboxstuff}{fg=black,bg=white!80!green}
%\begin{beamercolorbox}[wd=1\textwidth,sep=1em]{coloredboxstuff}
%	\textcolor{red}{\textbf{OpenAI on GPT-2}}\\
%	``We've trained a large-scale unsupervised language model which generates coherent paragraphs of text, achieves state-of-the-art performance on many language modeling benchmarks, and performs rudimentary reading comprehension, machine translation, question answering, and summarization—all without task-specific training.''
%\end{beamercolorbox}

%\setbeamercolor{coloredboxstuff}{fg=black,bg=white!80!red}
%\begin{beamercolorbox}[wd=1\textwidth,sep=1em]{coloredboxstuff}
%	``Due to concerns about large language models being used to generate deceptive, biased, or abusive language at scale, we are only releasing a much smaller version of GPT-2 along with sampling code. We are not releasing the dataset, training code, or GPT-2 model weights.''
%\end{beamercolorbox}

%\begin{itemize}
%	\item[-] \textbf{PR Focus} - reporters were given early information 
%	\item[-] \textbf{Gatekeeping} - malicious uses were hypothesized and we have no way of testing  
%%	\item[-] \textbf{Dual use} - OpenAI did not discuss dual-use technology
%\end{itemize}

%\end{frame}
\section{Background}

\subsection{\texttt{CTEXT} database}

\section{Erroneous class}

\begin{frame}{Text classifier}
A transparent and very popular probability learning model that can be implemented efficiently\\
\medskip
The probability of a document $d$ being in class $c$, $P(c \mid d)$ is computed as:
\begin{equation*}
P(c \mid d) \propto P(c) \prod_{i = 1}^{m}P(t_i \mid c) 
\end{equation*}
and the class of a document $d$ is computed as:\\

\begin{equation*}
c_{MAP} = arg~max_{c \in \{c_1, c_2 \}} P(c \mid d)
\end{equation*}

\end{frame}

\begin{frame}
	\begin{figure}
		\centering
		\includegraphics[width=0.75\textwidth]{/home/knielbo/Documents/figures/covariance_doc_misclass.png}
		\caption{some text here}
	\end{figure}
\end{frame}


\begin{frame}
\begin{figure}
	\centering
	\includegraphics[width=\textwidth]{/home/knielbo/Documents/figures/mclss_kld.png}
	\caption{Average distance to error class is shorter}
\end{figure}
\begin{equation}
D_{KL}(d_1 \mid d_2)  = \sum_{i = 1}^{n} d_{i1} \times \log_2 \frac{d_{i1}}{d_{i2}}
\end{equation}
\end{frame}


\begin{frame}
\begin{figure}
	\centering
	\includegraphics[width=\textwidth]{/home/knielbo/Documents/figures/mclss_mi.png}
	\caption{}
\end{figure}
\end{frame}


\section{Disruptive dynamics}
\begin{frame}
\begin{figure}
	\centering
	\includegraphics[width=.75\textwidth]{/home/knielbo/Documents/figures/shangshu_kld.png}
	\caption{Length normalized lexical density over time for the Shangshu}
\end{figure}
\end{frame}



\begin{frame}
\begin{figure}
	\centering
	\includegraphics[width=.75\textwidth]{../fig/diverge_matrix.png}
	\caption{}
\end{figure}
\end{frame}

\begin{frame}
\begin{figure}
	\centering
	\includegraphics[width=\textwidth]{../fig/signals.png}
	\caption{}
\end{figure}
\begin{figure}
	\centering
	\includegraphics[width=.35\textwidth]{../fig/novelty_resonance.png}
	\caption{}
\end{figure}
\end{frame}


\section{Persistent dynamics}


\section{Concluding remarks}




